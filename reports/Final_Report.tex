\newcommand*{\PathToOutput}{../_output}
\newcommand*{\PathToAssets}{../assets}
\newcommand*{\PathToBibFile}{bibliography.bib}
\providecommand{\mypath}{../reports/}
\def\bibliopath{../reports/}

\documentclass[12pt]{article}
\usepackage{graphicx}
\usepackage{amsmath}
\usepackage{float}
\usepackage{pdflscape}
\usepackage{subfigure}
\usepackage{caption}
\usepackage{subcaption}
\usepackage{rotating}
\usepackage{booktabs} 
\usepackage{adjustbox}
\usepackage{siunitx} 
\usepackage[a4paper, margin=1in]{geometry} 
\usepackage{natbib} 
\usepackage{placeins} 

\begin{document}
\title{
Investment Shocks and The Commodity Basis Spread}


\author{Duncan Park and Kyle Parran}
\begin{titlepage}
\maketitle


\begin{abstract}
In this study, we aimed to replicate the results from Table 1 of \cite{Yang2013}, which examines the effect of investment shocks on commodity basis spreads.
Using commodity futures data sourced from LSEG via the WRDS (Wharton Research Data Services) platform, we conducted rigorous statistical calculations and analysis in an attempt to validate the original findings.
Our replication effort involved data fetching, a detailed pre-processing stage, and an extensive analysis phase.
The entire process, including our methodology, is documented and available in a public GitHub repository.
This repository is automated end-to-end using py-doit, allowing users with WRDS access to replicate our results with ease.
\end{abstract}

\end{titlepage}

\section{Introduction}
\textit{``Intermediary Asset Pricing: New Evidence from Many Asset Classes''} \cite{He2017}, which explores the role of financial intermediaries in asset pricing, is an oft-cited paper that covers a range of asset classes that we have explored in this course.
Given the interests of us authors, we have chosen to replicate the results of \cite{Yang2013}, which focuses on the commodity asset class.
The paper examines the effect of investment shocks on the commodity basis spread, which is the difference between the futures price and the spot price of a commodity.
The visualization / data of focus in our work was Table 1 of \cite{Yang2013}, which presents the results of their analysis of basis spreads across many different commodities.

\newpage

\begin{table}[ht!]
  \centering
\begin{tabular}{l l c r r r r r}  % 8 columns total
\toprule
\textbf{Commodity} & \textbf{Symbol} & \textbf{N} 
    & \textbf{Basis} & \textbf{Freq.\ of bw.} 
    & $E[R^e]$ & $\sigma[R^e]$ & \textbf{Sharpe ratio} \\
\midrule
\multicolumn{8}{l}{\textbf{Agriculture}} \\
Barley        & WA & 235 & -3.66 & 27.66 & -0.24 & 19.62 & -1.21 \\
Butter        & 02 & 141 & -3.68 & 33.33 & 3.66  & 27.22 & 13.46 \\
Canola        & WC & 377 & -2.98 & 33.16 & -0.18 & 19.82 & -0.89 \\
Cocoa         & CC & 452 & -2.61 & 25.22 & 4.52  & 30.32 & 14.90 \\
Coffee        & KC & 420 & -2.57 & 36.90 & 6.00  & 36.52 & 16.44 \\
Corn          & C- & 468 & -6.03 & 23.08 & -0.01 & 23.35 & -0.04 \\
Cotton        & CT & 452 & -1.75 & 36.50 & 3.60  & 22.96 & 15.69 \\
Lumber        & LB & 468 & -5.63 & 33.55 & -1.13 & 22.80 & -4.98 \\
Oats          & O- & 468 & -5.65 & 31.20 & 0.44  & 28.90 & 1.53  \\
Orange juice  & JO & 448 & -3.08 & 36.61 & 2.32  & 29.56 & 7.86  \\
Rough rice    & RR & 265 & -7.56 & 26.04 & -1.50 & 25.01 & -6.01 \\
Soybean meal  & SM & 468 &  0.20 & 44.87 & 7.80  & 28.63 & 27.25 \\
Soybeans      & S- & 468 & -0.58 & 37.18 & 5.99  & 26.25 & 22.81 \\
Wheat         & W- & 468 & -2.88 & 38.68 & 2.79  & 23.76 & 11.72 \\
\midrule
\multicolumn{8}{l}{\textbf{Energy}} \\
Crude oil     & CL & 295 &  4.25 & 66.78 & 10.56 & 27.87 & 37.89 \\
Gasoline      & RB & 275 &  8.09 & 70.91 & 12.82 & 30.18 & 42.47 \\
Heating oil   & HO & 345 &  1.49 & 55.65 &  9.50 & 28.65 & 33.15 \\
Natural gas   & NG & 216 & -3.63 & 43.06 &  8.66 & 34.63 & 25.00 \\
Propane       & PN & 247 &  5.53 & 55.47 & 14.28 & 34.18 & 41.77 \\
Unleaded gas  & HU & 250 &  8.62 & 71.20 & 16.02 & 29.24 & 54.78 \\
\midrule
\multicolumn{8}{l}{\textbf{Livestock}} \\
Broilers      & BR &  19 &  4.58 & 52.63 &  1.49 &  7.28 & 20.53 \\
Feeder cattle & FC & 443 &  0.35 & 53.27 &  4.43 & 14.28 & 31.01 \\
Lean hogs     & LH & 468 &  2.66 & 59.40 &  7.98 & 22.34 & 35.70 \\
Live cattle   & LC & 468 &  0.46 & 50.64 &  4.55 & 14.92 & 30.46 \\
\midrule
\multicolumn{8}{l}{\textbf{Metals}} \\
Aluminum      & AL & 215 &  1.06 & 35.35 &  5.46 & 19.11 & 28.56 \\
Coal          & QL &  85 & -1.55 & 34.12 &  6.20 & 30.02 & 20.65 \\
Copper        & HG & 412 &  0.52 & 41.75 &  4.62 & 25.50 & 18.12 \\
Gold          & GC & 400 & -6.24 &  0.00 &  0.43 & 19.88 &  2.18 \\
Palladium     & PA & 362 & -2.16 & 30.66 & 10.21 & 35.19 & 29.01 \\
Platinum      & PL & 410 & -3.21 & 23.66 &  3.69 & 27.81 & 13.27 \\
Silver        & SI & 419 & -6.51 &  1.19 &  0.44 & 32.09 &  1.37 \\
\bottomrule
\end{tabular}

  \caption{Table 1 of \cite{Yang2013}}
\end{table}

\newpage

\newpage
\section{Methodology and Data Exploration}

Vitae congue mauris rhoncus aenean. Turpis egestas pretium aenean pharetra.
Non pulvinar neque laoreet suspendisse interdum consectetur libero id
faucibus. Id porta nibh venenatis cras sed. Viverra tellus in hac habitasse
platea. Sit amet facilisis magna etiam tempor orci eu lobortis elementum.
Porttitor leo a diam sollicitudin. Imperdiet proin fermentum leo vel orci
porta non. Maecenas pharetra convallis posuere morbi. Vel risus commodo
viverra maecenas accumsan lacus vel facilisis volutpat. Faucibus vitae
aliquet nec ullamcorper sit amet risus nullam. Sit amet venenatis urna cursus
eget nunc scelerisque viverra mauris. A arcu cursus vitae congue. Ullamcorper
morbi tincidunt ornare massa eget.


\begin{table}[ht!]
    \caption{Summary Stats}
    \centering
    \renewcommand{\arraystretch}{1.2}  
    \setlength{\tabcolsep}{5pt}        
    \begin{adjustbox}{max width=\textwidth}
    \input{\PathToOutput/sector_settlement_summary.tex}
    \end{adjustbox}
    \label{table:sector_settlement_summary}
\end{table}

Basis calculation was done like this formula:

\begin{equation}
    B_{i,t} \;=\; \frac{\ln\bigl(F_{i,t,T_1}\bigr)\;-\;\ln\bigl(F_{i,t,T_2}\bigr)}{\,T_2 \;-\; T_1\,}
  \label{eq:basis_formula}
\end{equation}

Excess return was done like this formula:

\begin{equation}
  R^e_{i,t+1,T} \;=\; \frac{F_{i,t+1,T}}{F_{i,t,T}} \;-\; 1
  \label{eq:return_formula}
\end{equation}


\newpage
\section{Replication Results}

Dictum at tempor commodo ullamcorper a. Nunc eget lorem dolor sed
viverra. Pellentesque pulvinar pellentesque habitant morbi tristique senectus
et netus et. Feugiat pretium nibh ipsum consequat nisl vel. Magna fermentum
iaculis eu non diam phasellus vestibulum. Et netus et malesuada fames ac
turpis. Purus ut faucibus pulvinar elementum integer enim neque. Sem
fringilla ut morbi tincidunt augue. Facilisi cras fermentum odio eu feugiat
pretium nibh. Pellentesque nec nam aliquam sem et. Mi bibendum neque egestas
congue quisque egestas diam.


\begin{table}[ht!]
    \caption{Table 1 Replication (Paper Time Period)}
    \centering
    \begin{adjustbox}{max width=\textwidth}
    \input{\PathToOutput/paper_table1_replication_paper.tex}
    \end{adjustbox}
    \caption*{
      Table 3 replicates the original analysis from Table 1 for the paper's time period, 
      summarizing each commodity’s performance across agriculture, energy, livestock, and metals. 
      Columns highlight sample size, basis, expected return, volatility, and Sharpe ratio, providing 
      an overview of both prospective risk and return.
    }
    \label{table:paper_table1_replication_paper}
\end{table}

\begin{table}[ht!]
  \caption{Table 1 Replication (Current Time Period)}
  \centering
  \begin{adjustbox}{max width=\textwidth}
  \input{\PathToOutput/paper_table1_replication_current.tex}
  \end{adjustbox}
  \caption*{
    Table 4 takes Table 3's approach and modifies the analyzed period to current, which would be from the end of the paper'S
    analysis to 02-28-25. As seen, there are noticeable changes in almost all categories for all commodities. The Sample's Sharpe ratio
    varies much more than in table 3 with ratios from 127 for Western barley and -137 for Coal. This highlights the effect of a smaller sample
    as well as the inconsistency of the return/risk metrics.
  }
  \label{table:paper_table1_replication_current}
\end{table}

\newpage
\section{Further Analysis}
In this section, we present several visual explorations of the commodity dataset. We begin by illustrating 
the coverage of monthly observations across products, noting any key gaps in the data. Next, we examine 
how different commodities correlate with one another, shedding light on which products tend to move together 
or diverge. Finally, we review long-term trends in monthly settlement prices, highlighting notable fluctuations and 
patterns in specific commodities. 



\begin{figure}[ht!]
  \centering
  \caption{Coverage Heatmap}
  \begin{adjustbox}{max width=\textwidth}
  \includegraphics[width=0.95\linewidth]{\PathToOutput/commodity_coverage_heatmap.png}
  \end{adjustbox}
  \caption*{Coverage heatmap for monthly observations across various commodities. 
  Each row corresponds to a specific commodity (with its symbol), and each column 
  indicates a monthly period. Dark squares signify the presence of data for that commodity-month 
  combination, while light squares show missing data. Notably, there is no data for Broilers (BR) 
  and a pronounced gap around 2009, illustrating non-uniform coverage across the dataset.}
  \label{fig:commodity_coverage_heatmap}
\end{figure}


\begin{figure}[ht!]
  \centering
  \caption{Correlation Heatmap}
  \begin{adjustbox}{max width=\textwidth}
  \includegraphics[width=0.95\linewidth]{\PathToOutput/commodity_correlation_heatmap.png}
  \end{adjustbox}
  \caption*{Pairwise correlation matrix of monthly settlement prices for the included commodities. Commodities 
  with insufficient data have been excluded to ensure reliable estimates. While most pairs exhibit moderate 
  positive correlations, several (e.g., silver and natural gas versus butter) stand out with notably negative values.}
\end{figure}

\clearpage
\begin{figure}[ht!]
  \centering
  \caption{Time-Series: Settlement Prices}
  \begin{adjustbox}{max width=\textwidth}
  \includegraphics[width=0.95\linewidth]{\PathToOutput/all_commodities_settlement.png}
  \end{adjustbox}
  \caption*{Time series of monthly settlement prices (1970–2025) for all included commodities. 
  Certain products, such as aluminum, exhibit data gaps in WRDS coverage. Cocoa and aluminum 
  stand out for their notably higher price fluctuations over this period, underscoring elevated 
  volatility relative to other commodities.}
  \label{fig:all_commodities_settlement}
\end{figure}  

\FloatBarrier

In analyzing coverage patterns across a diverse set of commodities, we identified notable gaps 
for certain products, as well as irregular coverage around specific periods like 2009. By examining 
correlation heatmaps, we observed that most commodities tend to move in moderate unison, though a few pairs 
exhibit distinctly negative relationships. The monthly settlement price data further highlighted that certain 
markets have experienced significantly higher fluctuations over time. Overall, Yang's paper yields an interesting 
strategy surrounding commodities, but without complete replication, and more in-depth knowledge of how he constructed
this table, it is hard to validate his approach and findings. 

\clearpage


\bibliographystyle{jpe}
\bibliography{\bibliopath references}

\end{document}
